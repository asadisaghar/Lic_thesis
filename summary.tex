%I need 12000 words~27 pages of introduction to my Licentiate as the literature review part, I think the space budget can go (roughly) like this:

%1%. Theory of structure formation 
% 5 pages (2 pages done)
%2%. Dark matter problem in general (including what particle physics has to say about DM particles)
% 5 pages
%3%. Small--scale dark matter in the context of LCDM vs. observations 
% 7 pages (~3 pages done)
%4%. Gravitational lensing formalism
% 3 pages
%5%. Optical vs. radio data in the context of our work + Radio interferometry
% 3 pages
%6%. Quasars vs. SMGs in the context of our work
% 3 pages

%----------------------------------------------------------------------------------------
%    PACKAGES AND OTHER DOCUMENT CONFIGURATIONS
%----------------------------------------------------------------------------------------

\documentclass[paper=a4, fontsize=11pt]{scrartcl} % A4 paper and 11pt font size
\oddsidemargin=-0.54cm
\evensidemargin=-0.54cm
\topmargin=-1.2cm
\textwidth=17cm
\textheight=25cm
\usepackage[T1]{fontenc} % Use 8-bit encoding that has 256 glyphs
\usepackage{fourier} % Use the Adobe Utopia font for the document - comment this line to return to the LaTeX default
\usepackage[english]{babel} % English language/hyphenation
\usepackage{amsmath,amsfonts,amsthm} % Math packages
\usepackage{hyperref}
\makeatletter
\newcommand\urlfootnote@[1]{\footnote{\url@{#1}}}
\DeclareRobustCommand{\urlfootnote}{\hyper@normalise\urlfootnote@}
\makeatother
\usepackage{graphicx}
\usepackage{caption}
\usepackage{subcaption}
\usepackage{float}
\usepackage{epstopdf}
\usepackage{titlesec}
\usepackage{verbatim}
\usepackage{amssymb}
\usepackage{amsmath}

\makeatletter
\@addtoreset{section}{part}
\makeatother
\usepackage{fancyhdr} % Custom headers and footers
\pagestyle{fancyplain} % Makes all pages in the document conform to the custom headers and footers
\fancyhead{} % No page header - if you want one, create it in the same way as the footers below
\fancyfoot[L]{} % Empty left footer
\fancyfoot[C]{} % Empty center footer
\fancyfoot[R]{\thepage} % Page numbering for right footer
\renewcommand{\headrulewidth}{0pt} % Remove header underlines
\renewcommand{\footrulewidth}{0pt} % Remove footer underlines
\setlength{\headheight}{13.6pt} % Customize the height of the header

\numberwithin{equation}{section} % Number equations within sections (i.e. 1.1, 1.2, 2.1, 2.2 instead of 1, 2, 3, 4)
\numberwithin{figure}{section} % Number figures within sections (i.e. 1.1, 1.2, 2.1, 2.2 instead of 1, 2, 3, 4)
\numberwithin{table}{section} % Number tables within sections (i.e. 1.1, 1.2, 2.1, 2.2 instead of 1, 2, 3, 4)

\setlength\parindent{0pt} % Removes all indentation from paragraphs - comment this line for an assignment with lots of text
%----------------------------------------------------------------------------------------
%    TITLE SECTION
%----------------------------------------------------------------------------------------

\newcommand{\horrule}[1]{\rule{\linewidth}{#1}} % Create horizontal rule command with 1 argument of height
\newcommand{\ignore}[1]{}

\title{    
\normalfont \normalsize 
%\textsc{Dark Cosmology Center\\Copenhagen - Denmark} \\ [25pt] % Your university, school and/or department name(s)
\horrule{0.5pt} \\[0.4cm] % Thin top horizontal rule
\huge  Gravitational lensing and radio interferometers probe sub-galactic dark matter structure\\
\large (story of my life) % The assignment title
\horrule{2pt} \\[0.5cm] % Thick bottom horizontal rule
}

\author{Saghar Asadi} % Your name

\date{\normalsize 2012--2014 \\ Department of Astronomy \\ Stockholm University} % Today's date or a custom date

\pagestyle{empty}

\begin{document}

\maketitle % Print the title
\newpage
\tableofcontents
\newpage

\begin{abstract}
Blah Blah Abstract
\end{abstract}

\newpage
\section*{Dark matter}
Dark matter as the second dominant component of our Universe is thought to be made of non--baryonic, cold (i.e. non--relativistic), weakly interacting massive particles (WIMPs). Some of these characteristics are inferred from cosmological evidence. For instance, observation of small--scale cosmic structure indicates that the dark matter must be non--relativistic. While the measured abundance of light elements in the Universe along with the Big Bang Nucleosynthesis (BNN) results in the dark matter being non--baryonic. The ``missing matter" in the Universe has presented itself both locally, i.e. in the galactic disk and around the Sun \citep[][]{}, and at the cosmic scale in the past century and even though the problem has been approached from the particle--physics point of view in addition to the cosmological one, it is still considered as one of the biggest challenges of both! 

I start this chapter by briefing independent cosmological evidence for the presence of dark matter (in the scale order rather than the historical order) and what each of them tell us about the nature and characteristics of dark matter as we know. The current state of the dark matter following by the detection methods currently used to approach the issue come next. In the end, I list some of the most competing alternative suggestion to the issue of the missing matter in the Universe.
 
\ignore{On the one hand, modern data suggests an insignificant amount of dark matter in the Solar vicinity which is made of baryonic but not luminous matter such as faint stars or Jupiter--like objects. In short, according to our current understanding of the Universe, dark matter is the dominant matter component of the Universe.}
\section{Cosmological evidence}

\section{Independent measurement of te amount of baryons in the Universe}
The amount of baryons in the Universe could me measured with five independent methods; all of which result in a fraction less than 5\% of the total content of the Universe.
  \begin{enumerate}
  \item X-ray emission from galaxy clusters
  \item anisotropies in the angular power spectrum of the CMB -- relative height of the odd and even picks
  \item the Big Bang Neucleosynthesis (BBN) - the abundance of light elements; $D$, $^3He$, $^4He$, and $^7Li$, each of which is observable independent of the others \citep[for Planck data see ][]{Planck2014}
  \item Baryonic acoustic oscillations
  \item Quasar absorption lines
  \end{enumerate}
  
\subsection{CMB temperature fluctuations}
\ignore{One very convincing evidence for the presence and amount of dark matter in our Universe is from the fluctuations of the CMB. Quote the cosmological density fractions for dark matter, baryon, and the Hubble constant for the latest Planck paper.}
Structures in the Universe are believed to have grown from small fluctuations in the density field of the Universe after the era of recombination. Density fluctuations are of the same order as temperature fluctuations ({\bf cite!})\ignore{\bf why? look back in the structure formation part!}, therefore measuring temperature fluctuations of the CMB reveals the order of magnitude of density fluctuations in the early Universe. However, taking only the baryonic matter into account, these fluctuations are two small to give rise to any structure formation in our expanding Universe. While the required amount of matter in the Universe is $\sim 0.2 - 0.3$ in units of the critical cosmological density ({\bf cite?}), the big bang nucleosynthesis puts an upper limit of $\sim 0.04$ on the amount of baryonic matter ({\bf cite!}) supporting the non--baryonic nature of dark matter. Moreover, if dark matter is non--baryonic, its density fluctuations can start growing already at the radiation--dominated era while the growth of baryonic matter is damped by radiation, therefore this could explain small density fluctuations of the CMB.
The first non--baryonic candidate for dark matter particles was neutrino. However, simulations based on this model (Doroshkevich \& Shandarin 1978) indicated that neutrino--dominated dark matter cannot give rise to small--scale structure in the distribution of galaxies, due to the cut--off at the power--spectrum of these rapidly--moving particles. This also suggested that suitable candidates for dark matter particles not only need to be dissipationless, but also required to be much more massive than neutrinos (Blumenthal et al. 1982, Bond et al. 1982, Peebles, 1982). Therefore, the first numerical cosmological simulations based on cold dark matter (CDM) was performed by (Melott et al. 1983) which could represent the small structures of the Universe much more accurately.

The discrepancy between the expected amplitude of temperature fluctuations in the CMB data based on structure formation (density/temperature fluctuations of the order of $10^{-3}$) cannot be explained without a large amount of non--baryonic matter in the radiation--dominated era. Another strong argument supporting the presence of some matter in addition to the baryonic matter in the Universe is also coming from the CMB. ({\bf the first peak in baryonic acoustic oscillations})

\subsection{Gravitational lensing}
The mathematical framework of gravitational lensing is a commonly--accepted framework in astronomy for a long time. However, only after the general relativistic corrections to the Newtonian formalism of gravitational lensing, the dark matter problem arose in this context. Gravitational lensing, as an accurate and reliable gravitational mass measurement of the lens, confirmed the discrepancy between the gravitational and luminous mass both on galactic and galaxy cluster scales.

\subsection{Masses of galaxy clusters}
First DM evidence by Zwicky in 1930s, using the velocity dispersion of galaxies in the Coma cluster \ref{Zwicky1993}

In the beginning of 1930s, Zwicky ({\bf cite!}) was first to notice the order--of--magnitude difference in dynamical mass of the Coma cluster with the mass derived from luminosities of individual galaxies. The discrepancy was not recognized as an issue until after its persistence in various cases ({\bf cite!})\ignore{cite Kahn \& Woltjer 1959 about the Andromeda's negative redshift and masses of galaxy groups Holmberg 1937! or rather just focus on rotational curve discrepancies whose agreement needs a new population of stars tenfold the regular ones in the galaxy with very large M/L  Einasto et al. 1974 ans Ostriker et al. 1974}. The previously--unknown population of dark (or high mass--to--light--ratio) matter was proven to dominate the mass budget of the Universe. The mean density of the matter was shown to be $\sim 0.2$ of the critical density of the Universe and different models for the dark matter started appearing in the literature. Firstly, the missing matter was thought to be in faint stars or hot gas but both failed to explain the situation\ignore{cite and explain why! or maybe not as it's not really important now!!}.

\subsection{Galactic rotation curves}
In the end of the 20$^\mathrm{th}$ century, both optical(Rubin et al. 1978, 1980) and radio(Bosma 1978) data confirmed the flat rotational curves of galaxies at large radii providing important evidence for the presence of massive halos around galaxies. This was followed by new X--ray observations of hot gas in galaxy clusters confirming that the hot gas reservoir cannot explain the missing mass either. The X--ray data, additionally, provided new measurements of the velocity dispersions of galaxies in clusters, confirming the previous dynamical mass estimates(Forman et al. 1972, Gursky et al. 1972, Kellogg et al. 1973). 

DM discovery at galactic scales in 1970s by Ver Rubin et al. deriving the existence of DM from the flat rotation curves of disk galaxies. They show that the rotational velocity of galaxy at large radii can only remain flat if there is a mass component increasing with radius. The Keplerian velocity of a galaxy whose dominant mass component is the disk implies a gravitational force experiences by any particle of mass $m$ at radius $r$ as $F = \frac{G M m}{r^2} = m v^2/r$. Therefore, $v = sqrt(GM/r)$, decreasing as $r^{1/2}$. While the observed rotational curve of disk galaxies show $v \approx$ constant, which requires $M \propto r$.


\section{Current state of dark matter}
The commonly--accepted class of DM particles are Weakly Interacting Massive Particles (WIMPs). These particles have two main characteristics:
  \begin{enumerate}
  \item They are heavy- The observational upper and lower bounds for dark matter particles coms from different source. While the upper limit that is placed by undetection of MACHOs (Massive Astrophysical Compact Halo Objects), neither with the Kepler satellite \citep[41][]{Griest+2014}, nor with microlensing surveys \citep[42\&43][]{Alcock+1998, Yoo+2004}, the lower limit is far less constrained. These limits expect the DM particle mass to be between $10^\mathrm{-31} GeV (= 9\times 10^\mathrm{-89} M_\odot) < m < 2\times 10^\mathrm{48} GeV (= 2 \times 10^\mathrm{-9} M_\odot)$ \citep[47 for lower bound][ and 42 \& 43 for the upper bound]{Hu+2000, Alcock+1998, Yoo+2004}. Note the 80 orders of magnitude mass range. This is how much we know of the second most abundant component in our Universe!
  \item They interact \emph{only} via the weak force
  \item They are thought to be \emph{dissipationless} (see below), and \emph{collisionless}- There are two sets of evidence for the dark matter to be essentially collisionless; The infamous ``Bullet cluster'' \citep[17][]{Clowe+2006}, and the the deviation of dark matter halos of galaxies and galaxy clusters fro spheres. However, the cross section of dark matter, and therefore the mean free path of its self--interaction estimated to be extremely large. This is one of the reasons that the presence of DM cores in the DM halo of dwarf galaxies presents a challenge to the commonly--accepted CDM model of dark matter. For, the cores are only explained by the self--collision of DM particles in the cuspy center of the halo and this requires collision cross sections much smaller than the estimated value \citep[49][]{Zavala+2013}, and therefore would be categorized as self--interacting DM. (see point 6 in page 9 of the lecture notes)
  \item They are non--relativistic- This rules out hot dark matter, and leave the cold and warm dark matter as remaining candidates. HDM particles are relativistic at the ``freeze--out'' epoch and therefore small perturbations do not survive, i.e. the early structures form in the Universe are supercluster. Smaller structures such as galaxy cluster, galaxies and dwarf galaxies form via fragmentation, but this does not match our observations. However, if dark matter particles are as described by CDM or WDM, structure formation follows a bottom--up hierarchy where small scale structures (down to dwarf galaxies in case of WDM and even smaller fluctuations in case of CDM survive). These small structures then merge/fall into the potential wells of larger structures through tidal striping. However, since tidal stripping is not 100\% efficient, a large number of small--scale structures are expected to await discovery. One challenge with our current model for DM and structure formation lies at this scale, where the number of dwarf galaxy--sized structures forming in pure CDM simulations is about an order of magnitude larger than the luminous dwarves observed around the Milky Way and Andromeda, while pure WDM simulations fail to reproduce the abundance of observed dwarf galaxies in the local Universe.
  \item {\bf (NOTE:``The DM is classified as hot, cold or warm according to how relativistic it is when galactic-size perturbations enter into the horizon (i.e. when these perturbations become encompassed by the growing horizon $\simeq ct$). This happens when $T \simeq keV$. Hot DM (HDM) is relativistic, Cold DM (CMD) is non-relativistic and Warm DM (WDM) is becoming non-relativistic at this moment (see e.g. [53]).''}
  \end{enumerate}
  
We have observed DM gravitational interaction with both ordinary and dark matter, but we have never observed [any evidence] of the decay of DM meaning that it must be stable or has a lifetime much larger than the age of the Universe. Since we have not been able to observe any evidence supporting the DM having any other type of interaction, but through gravity, makes it possible to wonder if there is no real entity as dark matter and all the evidence come from the lack of a good law of gravity. This lead to a category of solutions to the dark matter problem, the most successful of which is the Modified Newtonian Dynamics (MOND) \citep[14\&15][See erratum for 15!]{Milgrom1983, Bekestein2005}.

There are no observations showing the dark matter--photon interaction. Dark matter does not emit, absorb or reflect electromagnetic waves. However, DM particles and photons are thought to be able to have elastic interactions with each other. Although the observational upper limits on the cross section of these collisions are $\leq 4\times 10^\mathrm{-33} cm^2(m/GeV)$ \citep[38][]{Boehm+2014}.

{\bf The most important property of dark matter that makes is different from baryonic matter and at the same time very important in structure formation is a direct consequence of the small interaction/large cross section of DM particles with photons and it is that dark matter is \emph{dissipationless}, i.e. it does not cool by emitting photons and therefore, more massive structures could be formed earlier in the Universe by dark matter, that could not be formed by baryonic matter. Consequently, one important observational effect of this difference between dark and luminous matter is the presence of extended dark matter halos around cosmic structures from dwarf galaxies, to galaxies, galaxy groups and cluster of galaxies. While both dark matter and baryons experience the gravitational collapse, baryons emit photons while falling into the potential well of the halo (virial theorem), preserves their angular momentum and form disks while dark matter particle remains in the spherical form of a halo. The practical meaning of DM being dissipationless in analytical description of [Newtonian] structure formation is that the pressure term due to dark matter particles is negligibly small.}

There is a lower limit for the temperature of the Universe during the radiation--dominated epoch, coming from both inflation and BBN scenarios (according to these two independent theories, the ``reheating temperature'' is $T_\mathrm{HR} \geq 4$ MeV \citep[10][]{Hannestad2004}. Also, this is the period where most dark matter particle candidates are produced.

\section{Detection methods}
%%TASI lectures on astrophysical probes of dark matter
\subsubsection*{Indirect detection of DM particles}
3 main processes to study, (a) \emph{pair--annihilation}, (b) \emph{decay of DM particles into SM ones}, (c) \emph{elastic scattering between DM and SM particles}, where SM means standard model particles, i.e. baryons!
Observing the rate of SM particle production places constraints on DM particles via processes (a) and (b).
Momentum (mass) of DM particles is set by observing energy scale of SM ``messengers'' via (c).
The type of SM particles produced in (a) is strongly dependent on details of DM particle model.

In the very early Universe, when matter and radiation was coupled, at those high temperatures, the particle interaction ($\Gamma$) rate was larger than the Hubble expansion rate (H), we call the point in time/temperature where $\Gamma \sim H$, ``freeze--out''. After this point, particle species just ``redshift'' whatever momentum and number density they had at this point. This point differs for various particle species, with different number densities and interaction cross sections. If the particle species freeze out while they are relativistic, they are called \emph{hot relics}, otherwise they are \emph{cold relics} because the freeze--out temperature is much lower than the mass of the particle. The details of DM particle model puts an upper limit as well as a lower limit to the mass of \emph{cold} dark matter particles. This lower limit corresponds to the mass of first structures that gravitationally collapsed in the Universe, and for a typical WIMP model is $\sim M_\oplus \simeq 10^{-6} M_\odot$ ({\bf cite!}). However, this cut--off limit changes significantly in various models with substantial consequences for the dark matter predictions on small scales! 

\subsubsection{self--annihilating dark matter and gamma--ray emission}
Despite all the effort for direct detection of dark matter particles, the nature of these particles is still unknown. However, there are recent claims of a double--peak gamma--ray spectrum detections with the Fermi satellite (LAT), both in the center of our galaxy and nearby galaxy clusters (Weniger 2012, Tempel et al. 2012a, Hektor et al, 2013) which is interpreted as a signal of dark matter annihilation. ({\bf consider elaborating more on this and self--annihilating dark matter in general!})

Apart from gamma--ray hints of DM from the galactic center, the diffuse isotropic gamma--ray background (IGRB) at the new Fermi IGRB measurements could also place constraints on $M_\mathrm{min}$ of dark matter halos. ``A more detailed analysis of the IGRB, with new Fermi IGRB measurements and modeling of astrophysical backgrounds, may be able to probe values of $M_\mathrm{min}$ up to $\sim 1 M_\odot$; for the 130 GeV candidate and $\sim 10^{-6} M_\odot$; for the light DM candidates. Increasing the substructure content of halos by a reasonable amount would further improve these constraints.'' %\bibitem[Ng et al.(2014)]{2014PhRvD..89h3001N} Ng, K.~C.~Y., Laha, R., Campbell, S., et al.\ 2014, \prd, 89, 083001 

\section{Alternative solutions}
The relativistic modification of gravity came around after the Newtonian theory of gravity failed to predict correct position of Mercury in its orbit. A few decades later, and not even general relativity could estimate the correct dynamics for the galaxies without including a large amount of a mysterious invisible matter, soon called the ``dark matter''. Even today, around 80 years after Zwicky first mentioned the discrepancy, we have no clue what dark matter is and if this is the right approach at all, in comparison to further modification in our theory of gravity. 
%%%%%%%%%%
Therefor, another appealing idea, suggested by (Milgrom \& Bekenstein 1987) was questioning the validity of the Newtonian law of gravity in large distances. This theory which is now know as modified Newtonian dynamics (MOND) has been able to explain a number of observational data without any dark matter. However, there are still a number of observational evidence in favor of non--baryonic dark matter which is inexplicable with MOND. 


MOND was first introduced to solve the discrepancy with the flat rotation curves of galaxies through a general form of the infamous $F = ma$, which only deviates from the classical form in systems with very small accelerations. The law of gravity according to MOND is $F=\mu(a)ma$. Therefore, for a particle of mass $m$ moving around a massive galaxy of $M$ at radius $r$ we have: $F = \frac{GMm}{r^2} = \mu m a$. For very small accelerations, i.e. $a<<a_0$, $\mu = a/a_0$, and therefore at large radii from the center of a galaxy, where $a<<a_0$, $a=\sqrt(GMa_0)/r$. On the other hand, $a = v^2/r$. Combining the two, $v = (GMa_0)^\mathrm{1/4}$, and the rotational curve of the galaxy becomes flat consistent to observations. However, it fails at explaining certain phenomena such as the bullet cluster \citep[17][]{Clowe+2006}, and other large--scale observations \citep[16][]{Famaey+2012}.

%http://www.quantumdiaries.org/2013/06/26/does-dark-matter-really-exist/
%Einasto 2013 -- Profumo 2012 -- Popolo 2013 -- Profumo 2013

\begin{thebibliography}{99}
\bibitem[\protect\citeauthoryear{Alcock et al.}{1998}]{Alcock+1998} 
Alcock C., et al., 1998, ApJ, 499, L9 
\bibitem[\protect\citeauthoryear{B{\oe}hm et 
al.}{2014}]{Boehm+2014} B{\oe}hm C., Schewtschenko J.~A., 
Wilkinson R.~J., Baugh C.~M., Pascoli S., 2014, MNRAS, 445, L31 
\bibitem[\protect\citeauthoryear{Bekenstein}{2004}]{Bekenstein2004} 
Bekenstein J.~D., 2004, PhRvD, 70, 083509 
\bibitem[\protect\citeauthoryear{Clowe et al.}{2006}]{Clowe+2006} 
Clowe D., Brada{\v c} M., Gonzalez A.~H., Markevitch M., Randall S.~W., 
Jones C., Zaritsky D., 2006, ApJ, 648, L109 
\bibitem[\protect\citeauthoryear{Griest, Cieplak, 
\& Lehner}{2014}]{Griest+2014} Griest K., Cieplak A.~M., Lehner M.~J., 2014, ApJ, 786, 158 
\bibitem[\protect\citeauthoryear{Hannestad}{2004}]{Hannestad2004} 
Hannestad S., 2004, PhRvD, 70, 043506 
\bibitem[\protect\citeauthoryear{Hu, Barkana, 
\& Gruzinov}{2000}]{Hu2000} Hu W., Barkana R., Gruzinov A., 2000, PhRvL, 85, 1158 
\bibitem[\protect\citeauthoryear{McGaugh}{2015}]{2015CaJPh..93..250M} 
McGaugh S.~S., 2015, CaJPh, 93, 250 
\bibitem[\protect\citeauthoryear{Milgrom}{1983}]{Milgrom1983} 
Milgrom M., 1983, ApJ, 270, 365 
\bibitem[\protect\citeauthoryear{Planck Collaboration et 
al.}{2014}]{Planck2014} Planck Collaboration, et al., 2014, A\&A, 571, A16 
\bibitem[\protect\citeauthoryear{Yoo, Chanam{\'e}, 
\& Gould}{2004}]{2004ApJ...601..311Y} Yoo J., Chanam{\'e} J., Gould A., 2004, ApJ, 601, 311 
\bibitem[\protect\citeauthoryear{Zavala, Vogelsberger, 
\& Walker}{2013}]{Zavala+2013} Zavala J., Vogelsberger M., Walker M.~G., 2013, MNRAS, 431, L20 
\bibitem[\protect\citeauthoryear{Zwicky}{1933}]{Zwicky1933} Zwicky 
F., 1933, AcHPh, 6, 110 
\end{thebibliography}
\end{document}

